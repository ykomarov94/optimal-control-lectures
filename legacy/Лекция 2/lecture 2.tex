\documentclass[12pt, a4paper]{article}



%\usepackage[cp1251]{inputenc}
\usepackage{a4wide} % уменьшает поля
\textwidth=11.5cm

\usepackage[utf8]{inputenc}
\usepackage[russian]{babel} % включает русский язык
\usepackage{graphicx} % позволяет подключить .eps - файлы
\usepackage{amsmath}
\usepackage{amsthm} % теоремы от AMS
\usepackage{amssymb} % для работы с математическими R и проч.
\usepackage{floatrow}
\usepackage{mathrsfs}
\usepackage{mathtools} % содержит \coloneqq -- присванивание

\usepackage{accents}
\newcommand{\ubar}[1]{\underaccent{\bar}{#1}}

\usepackage[usenames]{color} % чтобы цвета в hyperref понимались
\usepackage{ulem} % дает зачеркнутый текст по \sout
\usepackage[unicode,
			breaklinks,colorlinks,
			linkcolor=Black,
			citecolor=Black,
			hyperfootnotes=false]
			{hyperref} % кликабельное оглавление без кликабельных футноутов


\newtheoremstyle{rusdef}
  {3pt}% measure of space to leave above the theorem. E.g.: 3pt
  {3pt}% measure of space to leave below the theorem. E.g.: 3pt
  {\itshape}% name of font to use in the body of the theorem
  {\parindent}% measure of space to indent
  {\bfseries}% name of head font
  {.}%
  {.5em}%
  {}
   

\theoremstyle{rusdef}
\newtheorem{define}{Определение} % определение по-русски
\newtheorem{theorem}{Теорема}
\newtheorem{prop}{Предложение}
\renewcommand\qedsymbol{$\blacksquare$}
\newtheorem{statement}{Утверждение}
\newtheorem{remark}{Замечание}
\newtheorem{lemma}{Лемма}
\newtheorem{corol}{Следствие}
\newtheorem{assume}{Предположение}
\newtheorem{example}{Пример}

\newcommand\abs[1]{\left\lvert #1 \right\rvert} % модуль
\newcommand\bracket[1]{\left( #1 \right)} % скобки
\newcommand\scalar[1]{\left < #1 \right >} % скалярное произведение
\newcommand{\R}{\ensuremath{\mathbb{R}}} % R - мн-во вещественных чисел
\newcommand{\N}{\ensuremath{\mathbb{N}}} % N - мн-во натуральных чисел
\newcommand{\X}{\mathscr{X}} % красивая Х для начального и конечного множеств
\renewcommand{\P}{\mathscr{P}} % красивая P для ограничений на управление
\newcommand{\then}{\Rightarrow}
\newcommand{\h}{\mathbb{h}}
\newcommand{\e}{\mathbf{e}}

\renewcommand{\H}{\mathcal{H}} % красивая H для Гамильтона-Понтрягина
\renewcommand{\d}{\partial} % чтобы долго не писать частную производную
\newcommand{\norm}[1]{\left\lVert #1 \right\rVert} % норма
\DeclareMathOperator*{\thus}{\Rightarrow} % следствие с возможностью использовать limits
\DeclareMathOperator*{\To}{\longrightarrow}
\DeclareMathOperator*{\Argmax}{Argmax} % Argmax с возмножностью использовать limits

\usepackage{indentfirst} % абзац после заголовка

\DeclareMathOperator{\sgn}{sgn} % сигнум
\DeclareMathOperator{\rank}{rank} % ранг

\begin{document}

\begin{center}
{\Huge \textbf{Теоремы о зависимости решения от начальных данных}}
\end{center}

Сегодня мы затронем вопросы зависимости решения по Каратеодори от начальных данных, а именно исследуем непрерывность и дифференцируемость.

Рассмотрим следющую задачу Коши:
\begin{equation}\label{eq:system}
\left\{
\begin{aligned}
& \dot{x}(t) = g(t, x(t)),\\
& x(t_0) = x^0,
\end{aligned}
\right.
\tag{ЗК}
\end{equation}
её решение будем обозначать$x[t] = x(t, t_0, x^0)$.

\textbf{Вопрос 1:} когда $x(t, t_0, x^0)$ непрерывна по $(t, t_0, x^0)$? \textit{Отдельно по $t$ решение $x[t]$ всегда непрерывно по определению решения \eqref{eq:system}}.

\textbf{Вопрос 2:} когда $x(t, t_0, x^0)$ дифференцируема по $x^0$, а $\dfrac{\d x}{\d x^0}(t, x, x^0)$ непрерывна по $(t, t_0, x^0)$.

Для линейных систем всё просто: если $\dot{x}(t) = A(t) x(t) + c(t)$, то
$$
x(t) = X(t, t_0) x^0 + \int\limits_{t_0}^{t} X(t,s) c(s) ds.
$$
Поскольку $X(t, t_0)$ непрерывна, ответ на первый вопрос <<да>>. А что с дифференцируемостью? Обозначим
$$
Y[t] = \dfrac{\d x(t)}{\d x^0} = X(t, t_0),
$$
решение этого матричного дифференциального уравнения:
$$
Y[t] = Y(t, t_0, x^0).
$$
Если мы его продифференцируем, получим
$$
\left\{
\begin{aligned}
& \dfrac{dY}{dt} = A(t) Y[t], \\
& Y[t_0] = E,
\end{aligned}
\right.
$$
то есть здесь всё снова следует из общего вида решения.

Для систем общего вида:
\begin{itemize}
\item Вопрос 1 --- ответ <<да>> (при накладываемых условиях на $g(t,x)$);
\item Вопрос 2 --- ответ <<да>> при условии, что существует $\exists \dfrac{\d g}{\d x}$ измерима по $t$ и непрерывна по $x$.
\end{itemize}

Продифференцируем $\dot{x}[t] = g(t, x[t])$ по $x^0$:
$$
\dfrac{\d}{\d x^0} \left( \dfrac{\d x(t, t_0, x^0)}{\d t} \right) = 
\dfrac{\d}{\d x^0} \left( g(t, x(t, t_0, x^0)) \right),
$$
$$
\dfrac{\d}{\d t} \left( \dfrac{\d x(t, t_0, x^0)}{\d x^0} \right) = 
\dfrac{\d g}{\d x} \frac{\d x}{\d x^0}.
$$
Вновь обозначив $Y[t] = \dfrac{\d x}{\d x^0}$, получим уравнение в вариациях:
$$
\dfrac{\d Y[t]}{\d t} = \dfrac{\d g(t, x[t])}{\d x} Y[t],
$$
начальное условие снова
$$
Y[t_0] = E.
$$

Помимо тех теорем, которые будут рассмотрены сегодня, существуют также и более общие, например, теорема Понтрягина о зависимости от параметра $\mu$, но они нам в курсе не понадобятся.

\section{Непрерывность}
%\begin{theorem}[О непрерывности]
%Пусть $g(t,x)$ определена при п.\,в. $\dot{\forall} t \in [\tau_0, \tau_1]$, $\tau_0 < t_0 < t_1 < \tau_1$ и для всех $\forall x \in B_r(x^0)$. Пусть, кроме того, $g(t,x)$ измерима по $t$ для всех $\forall x \in B_r(x^0)$ и непрерывна по $x$ для п.\,в. $\dot{\forall} t \in [\tau_0, \tau_1]$.
%
%Пусть $x[t] = x(t, t_0, x^0)$ --- решение \eqref{eq:system}, $t \in [\tilde{\tau_0}, \tilde{\tau_1}]$, $\tau_0 \leqslant \tilde{\tau}_0 < t_0 < \tilde{\tau}_1 \leqslant \tau_1$, а $g(t,x)$ дополнительно удовлетворяет условиям существования и единственности решения \eqref{eq:system} (без продолжимости).
%
%Тогда $\exists \delta > 0$ такое, что если $\tau \in (\tilde{\tau}_0, \tilde{\tau}_1)$ и $\xi \in U_{\delta}(x[\tau])$ (т.\,е. $\norm{x[\tau] - \xi} \leqslant \delta$), то $\exists !$ решение задачи Коши:
%$$
%\left\{
%\begin{aligned}
%& \fot{\tilde{x}}(t) = g(t, \tilde{x}(t)), \\
%& \tilde{x}[\tau] = \xi
%\end{aligned}
%\right.
%$$
%\end{theorem}

Мы сформулируем частный случай теоремы о непрерывной зависимости.

Будем предполагать, что
$$
g \colon [T_0, T_1] \times \R^n \to \R^n, \;\; -\infty \leqslant T_0 < T_1 \leqslant +\infty.
$$

Кроме того, мы будем накладывать на $g(t,x)$ следующие ограничения ($\star$):
\begin{enumerate}
\item $g(t,x)$ измерима по $t \in [T_0, T_1]$ для всех $\forall x \in \R^n$,\\
	  $g(t,x)$ непрерывна по $x$ для п.~в. $\dot{\forall} t \in [T_0, T_1]$;
\item $\exists L = const > 0$: $\norm{g(t, x') - g(t, x'')} \leqslant L \norm{x' - x''}$ для всех $\forall x', x'' \in \R^n$ и п.~в. $\dot{\forall} t \in [T_0, T_1]$;
\item $\exists A, B = const, A > 0, B > 0$:\\
	  $\norm{g(t,x)} \leqslant A \norm{x} + B$ для всех $\forall x \in \R^n$ и п.в. $\dot{\forall} t \in [T_0, T_1]$.
\end{enumerate}

Пусть $T_0 < \tau < T_1$. Мы будем рассмтривать чуть более общий вид системы:
\begin{align}
&\dot{x}(t) = g(t, x(t)), \label{eq:cont_eq1} \\
&x(\tau) = \xi, \label{eq:cont_eq2}
\end{align}
где $\tau \in (T_0, T_1)$.

\begin{define}
Функция $y(\cdot)$ называется $\varepsilon$-решением \eqref{eq:cont_eq1}, если $y(\cdot) \in AC([\tau_0, \tau_1, \R^n])$ и $\norm{\dot{y}(t) - g(t, y(t))} \leqslant \varepsilon$ для п.\,в. $\dot{\forall} t \in [\tau_0, \tau_1]$.
\end{define}

\begin{lemma}
Пусть $y^1(\cdot)$ --- $\varepsilon_1$-решение \eqref{eq:cont_eq1}, а $y^2(\cdot)$ --- $\varepsilon_2$-решение \eqref{eq:cont_eq1}, при этом $\varepsilon_1, \varepsilon_2 \geqslant 0$, $y^1, y^2$ определены на $[\tau_0, \tau_1]$, $T_0 < \tau_0 < \tau < \tau_1 < T_1$ и $\norm{y^1(\tau) - y^2(\tau)} \leqslant \delta$. Тогда 
$$
\norm{y^1(t) - y^2(t)} \leqslant \delta \e^{L \abs{t - \tau}} + \dfrac{\varepsilon}{L}\left( \e^{L \abs{t - \tau}} - 1 \right),
$$
где $\varepsilon = \varepsilon_1 + \varepsilon_2 \geqslant 0$, $\tau_0 < t < \tau_1$.
\end{lemma}
\begin{proof}
Рассмотрим две функции
$$
z^j (t) = \dot{y}^j (t) - g(t, y^j (t));
$$
$\norm{z^j (t)} \leqslant \varepsilon_j$ для п.\,в. $\dot{\forall} t$.

Очевидно, что $z^j(\cdot)$ --- измеримы. Имеем:
$$
y^j (t) - y^j(\tau) - \int\limits_{\tau}^{t} g(s, y^j(s)) ds = \int\limits_{\tau}^{t} z^j (s) ds.
$$

Вычтем одно равенство из другого:
$$
\left( y^1(t) - y^2(t) \right) - \left(y^1(\tau) - y^2(\tau)\right) -
$$
$$
- \int\limits_{\tau}^{t}\left[ g(s, y^1(s)) - g(s, y^2(s)) \right] ds = \int\limits_{\tau}^{t} \left[ z^1(s) - z^2(s) \right] ds.
$$

Обозначим $\Delta y(t) = y^1(t) - y^2(t)$, $r(t) = \norm{\Delta y(t)}$, тогда
\begin{equation}\label{eq:r_estimate}
r(t) \leqslant r(\tau) + L \int\limits_{\min(t,\tau)}^{\max(t,\tau)} r(s) ds + \varepsilon\abs{t - \tau}.
\end{equation}

Не ограничивая общности, будем предполагать $t > \tau$. Обозначим $\dot{R}(t) = r(t)$, тогда, в силу условия $r(\tau) \leqslant \delta$, получим:
$$
\dot{R}(t) - LR(t) \leqslant \delta + \varepsilon (t - \tau).
$$
Домножая на $\e^{-Lt}$, получим
$$
\dfrac{d}{dt} \left( R \e^{-Lt} \right) \leqslant (\delta + \varepsilon(t - \tau))\e^{-Lt}.
$$
Проинтегрировав на отрезке времени от $\tau$ до $t$, получим
$$
R(t)\e^{-Lt} - R(\tau) \e^{-L\tau} \leqslant \dfrac{\delta}{L} \left( \e^{L\tau} - \e^{-Lt}\right) - \dfrac{\varepsilon \tau}{L} \left( \e^{-L\tau} - \e^{-Lt} \right) +
$$
$$
\dfrac{\varepsilon}{L} \left( \tau \e^{-L\tau} - t\e^{-Lt} \right) + \dfrac{\varepsilon}{L^2} \left( \e^{-L\tau} - \e^{-Lt} \right).
$$
Домножим всё на $\e^{Lt}$, получим:
$$
R(t) \leqslant \dfrac{\delta}{L} \left( \e^{L(t - \tau)} - 1 \right) - \dfrac{\varepsilon}{L}(t - \tau) + \dfrac{\varepsilon}{L^2} \left( \e^{L(t - \tau)} - 1 \right).
$$
Это неравенство можно подставить в оценку \eqref{eq:r_estimate} (за $R(t)$ мы обозначили интеграл, стоящий справа от знака неравенства):
$$
r(t) \leqslant \delta + \delta \left( \e^{L(t - \tau)} - 1 \right) + \varepsilon \left( \dfrac{\e^{L(t - \tau} - 1}{L} - (t - \tau) + (t - \tau) \right).
$$

Приводя слагаемые, получим оценку из условия леммы.
\end{proof}

Рассмотрим $x(t, \tau, \xi)$ --- решение \eqref{eq:cont_eq1}, \eqref{eq:cont_eq2}.
\begin{theorem}
Пусть выполнены условия ($\star$). Тогда в области $(T_0, T_1)^2 \times \R^n$ решение $x(t, \tau, \xi)$ непрерывно.
\end{theorem}
\begin{proof}
Обозначим $V = (T_0, T_1) ^ 2 \times \R^n$. Рассмотрим последовательность:
$$
x^{(0)}(t,\tau,\xi) = y[t] + \xi - y[\tau],
$$
где $y[t] = x(t, t_0, x^0)$ для некоторого $t_0$: $T_0 < t_0 < T_1$.
$$
y[t] = y[\tau] + \int\limits_{\tau}^{t} g(s, y[s]) ds.
$$

Для $k \in \N$ определим
$$
x^{(k)}(t, \tau, \xi) = \xi + \int\limits_{\tau}^{t} g(s, x^{(k-1)}(s, \tau, \xi)) ds.
$$

Тогда
$$
\norm{x^{(0)} (t, \tau, \xi) - y[t]} = \norm{\xi - y[\tau]},
$$
$$
\norm{x^{(1)}(t, \tau, \xi) - x^{(0)}(t, \tau, \xi)} =
$$
$$
= \norm{\int\limits_{\tau}^{t} \left[ g\left(s, x^{(0)}(s, \tau, \xi)\right) - g(s, y[s])) \right]} ds \leqslant
$$
$$
\leqslant L \norm{\xi - y[\tau]} \abs{t - \tau}.
$$

Для произвольного $k \in \N$:
$$
\norm{ x^{(k+1)} (t, \tau, \xi) - x^{(k)} (t, \tau, \xi) } \leqslant \dfrac{L^ {k+1} \abs{t - \tau}^{k+1}}{(k+1)!} \norm{\xi - y[\tau]},
$$
т.~е.
$$
\norm{ x^{(k+1)} (t, \tau, \xi) - x^{(k)} (t, \tau, \xi) } \to 0
$$
при $k \to \infty$, причём равномерно на любом $\forall K$ --- компакте из $V$. Значит, $x^{(k)}$ равномерно сходится на нём (в силу сходимости в $C$).

Тогда имеем:
$$
\norm{ x^{(k)}(t, \tau, \xi) - y[t]} \leqslant \norm{x^{(k)} (t, \tau, \xi) - x^{(k - 1)}(t, \tau, \xi)} +
$$
$$
+ \norm{ x^{(k-1)}(t, \tau, \xi) - y[t]} \leqslant \left\{ \text{ряд Тейлора} \right\} \leqslant \e^{L \abs{t - \tau}} \norm{\xi - y[\tau]}.
$$

Т.\,к. $x(t, \tau, \xi) = \xi + \int\limits_{\tau}^{t} g(s, x(s, \tau, \xi))ds$, то $$
x^{(k)}(t, \tau, \xi) \rightrightarrows x(t, \tau, \xi), \; k \to \infty
$$
и выполняется оценка
$$
\norm{x(t, \tau, \xi) - y[t]} \leqslant \e^{L \abs{t - \tau}} \norm{\xi - y[\tau]}.
$$

Из непрерывности $y[t]$ (в силу ($\star$)) следует непрерывность $x^{(k)}(t, \tau, \xi)$, а значит, и непрерывность $x(t, \tau, \xi)$.
\end{proof}

\begin{corol}
$x(t, \tau, \cdot)$ является гомеоморфизмом (топологическое отображение), т.~е. $x(t, \tau, \cdot) \colon \R^n \to \R^n$ взаимно однозначное, непрерывное и обратно непрерывное отображение:
$$
x^{-1}(t, \tau, \cdot) = x(\tau, t, \cdot).
$$
\end{corol}

\begin{corol}
$$
\left\{
\begin{aligned}
& \dot{x}(t) = f(t, x(t), u(t)), \\
& x(t_0) = x^0, \;\; u(t) \in \P(t).
\end{aligned}
\right.
$$

$\mathscr{X}[t] = \mathscr{X}(t, t_0, x^0)$. Тогда если $\xi \in int \mathscr{X}[\tau]$, то $\forall t > \tau$, $\forall u(\cdot)$ для $s \in [\tau, t]$
$$
x[s] = x(s, \tau, \xi \rvert u(\cdot)) \in int \mathscr{X}[s].
$$
\end{corol}

\begin{remark}
Если $\xi \in int \mathscr{X}[\tau]$, то некоторая окрестность $U_{\delta}(\xi) \subseteq \mathscr{X}[\tau]$. Возьмём $x \in U_{\delta}(\xi)$ и рассмотрим $y[s] = x(s, \tau, x \rvert u(\cdot))$ \textit{(управление $u(\cdot)$ такое же)}.
$$
\bigcup\limits_{x \in U_{\delta}(\xi)} \left\{ y[s] \right\} \subseteq \mathscr{X}[s]
\thus
x[s] \in int \bigcup\limits_{x \in U_{\delta}(\xi)} \left\{ y[s] \right\}.
$$
\end{remark}

\begin{remark}
Таким образом, если в нелинейной системе траектория движется по границе трубки достижимости и в какой-то момент <<проваливается>> внутрь неё, то снова оказаться на границе траектория не сможет. В соответствии с принципом максимума, такая траектория может уже не быть оптимальной, например, в задаче быстродействия. Отличие от линейных систем здесь в том, что для последних траектория уйти с границы не может.
\end{remark}

\section{Дифференцируемость}

\begin{theorem}
Пусть выполнены условия ($\star$) и, кроме того, производная $\dfrac{\d g}{\d x}$ существует для всех $\forall x \in \R^n, \; \dot{\forall} t \in [T_0, T_1]$, измерима по $t$, непрерывна по $x$, а также удовлетворяет следующему условию регулярности:\\
пусть для любых $\forall \tau_0, \tau_1$: $T_0 < \tau_0 < \tau_1 < T_1$, для любых $\forall \varepsilon >0$, $\forall D \subseteq \R^n$, $D$ --- непустой компакт, существует $\exists \delta(\varepsilon, \tau_0, \tau_1, D) > 0$:
$$
\forall x', x'' \in D \colon \norm{x' - x''} \leqslant \delta, \forall t \in [\tau_0, \tau_1] \thus
$$
$$
\norm{\dfrac{\d g}{\d x}(t, x') - \frac{\d g}{\d x}(t, x''))} \leqslant \varepsilon.
$$

Тогда $\exists \left. \dfrac{\d x(t, \tau, \xi))}{\d \xi} \right|_{(t, t_0, x^0)} = Y(t, \tau, \xi)$, $Y[\cdot] \in AC$, 
удовлетворяющая уравнению в вариациях
\begin{equation}\label{eq:variation_eq}
\dot{Y}[t] = \dfrac{\d g(t, x(t, \tau, \xi))}{\d x} Y[t],
\tag{УВ}
\end{equation}
$$
Y[\tau] = E.
$$
\end{theorem}
\begin{proof}
Обозначим $y[t] = x(t, \tau, \xi^0)$. Зафиксируем $\forall h \in \R^n$  и рассмотрим $x_h[t] = x(t, \tau, \xi^0 + \alpha h)$. Обозначим норму разности этих решений
$$
\vartheta_h(t, \tau, \xi^0, \alpha) = x_h[t] - y[t].
$$

Существует ли $\lim\limits_{\alpha \to 0} \dfrac{\vartheta_h(t, \tau, \xi^0, \alpha)}{\alpha}$ (на некотором компакте $\abs{\alpha} \leqslant \alpha_0$)?
Поскольку $y[t]$ и $x_h[t]$ являются решениями исходной системы, они также являются её $\varepsilon$-решениями с $\varepsilon = 0$. Мы можем применить лемму 1 для того, чтобы оценить $\vartheta_h$:
$$
\norm{\vartheta_h(t, \tau, \xi^0, \alpha)} \leqslant \abs{\alpha} \norm{h} \e^{L\abs{t - \tau}} \To\limits_{\alpha \to 0} 0,
$$
причём сходимость равномерна по $(t, \tau, \xi)$ в пределах произвольного компакта.

Теперь продифференцируем $\vartheta_h$ по времени. Для п.~в. $\dot{\forall} t$:
$$
\dfrac{d \vartheta_h(t, \tau, \xi^0, \alpha)}{dt} = g(t, x_h[t]) - g(t, y[t]) = 
$$
$$
= \left\{ \text{ф-ла конечных приращений} \right\} =
$$
$$
\left[ \dfrac{\d g}{\d x}(t, y[t] + \beta(t)(x_h[t] - y[t])) \right] \vartheta_h(t, \tau, \xi^0, \alpha) = **
$$

Пусть $\tau_0, \tau_1$: $T_0 < \tau_0 < \tau_1 < T_1$ и $D \in \R^n$, $D$ --- компакт, таковы, что $\forall t \in [\tau_0, \tau_1]$, $\forall \beta \in [0,1]$, $\forall \tau \in [\tau_0, \tau_1]$ и \\ $\forall \alpha \colon \abs{\alpha} \leqslant \alpha_0$ $\thus y[t] + \beta \vartheta_h(t, \tau, \xi^0, \alpha) \in D$. \textit{Это возможно ввиду ограниченности $y[t]$ и полученной оценки на $\vartheta_h$}.

Из условия регулярности на $g$:
$$
\forall \varepsilon > 0 \; \exists \delta = \delta(\varepsilon, \tau_0, \tau_1, D) > 0 \colon
$$
$$
\norm{\dfrac{\d g(t, y[t])}{\d x} - \dfrac{\d g(t, y[t] + \beta[t] \vartheta_h)}{\d x}} \leqslant \varepsilon \text{ при } \norm{\vartheta_h} \leqslant \delta.
$$
Обозначим разность производных, стоящих под знаком нормы, за $-\Gamma$.

Тогда
$$
** = \left[ \dfrac{\d g}{\d x}(t, y[t]) + \Gamma \right] \vartheta_h(t, \tau, \xi^0, \alpha) =
$$
$$
 = \left\{ \gamma \colon \alpha \gamma = \Gamma \vartheta_h \thus \norm{\gamma} \leqslant \norm{\Gamma} \norm{h} \e^{L \abs{t - \tau}} \right\} =
$$
$$
= \dfrac{\d g}{\d x}(t, y[t])\vartheta_h(t, \tau, \xi^0, \alpha) + \gamma \alpha.
$$

Введём новую функцию $\chi_h(t, \tau, \xi^0, \alpha) = \dfrac{\vartheta_h(t, \tau, \xi^0, \alpha)}{\alpha}$. 
$$
\dfrac{\d \chi}{\d t}(t, \tau, \xi^0, \alpha) = \dfrac{\d g}{\d x}(t, y[t])\chi_h(t, \tau, \xi^0, \alpha) + \gamma,
$$
т.~е. $\chi_h$ - это $\tilde{\varepsilon}$-решение уравнения \eqref{eq:variation_eq}
$$
\dot{z}[t] = \dfrac{\d g}{\d x} (t, y[t])z[t]
$$
с $\norm{\gamma} \leqslant \tilde{\varepsilon}$. Покажем, что это действительно так.

Пусть $\psi_h[\cdot] = \psi_h(\cdot, \tau)$ --- истинное решение \eqref{eq:variation_eq} с начальным условием $\psi_h[\tau] = h$.
Заметим, что
$$
\chi_h(\tau, \tau, \xi^0, \alpha) = \dfrac{x_h[\tau] - y[\tau]}{\alpha} = \dfrac{\xi^0 + \alpha h - \xi^0}{\alpha} = h,
$$
тогда
$$
\norm{\chi_h(\tau, \tau, \xi^0, \alpha) - \psi_h(\tau, \tau)} = 0.
$$
Применим лемму 1 к \eqref{eq:variation_eq}: при $\delta = 0$, $\varepsilon_1 = \tilde{\varepsilon}$, $\varepsilon_2 = 0$ (т.~е. $\varepsilon = \varepsilon_1 + \varepsilon_2 = \tilde{\varepsilon}$) справедлива оценка
$$
\norm{\chi_h(t, \tau, \xi^0, \alpha) - \psi_h(t, \tau)} \leqslant \dfrac{\tilde{\varepsilon}}{L} \left( \e^{L \abs{t - \tau}} - 1 \right) < \varepsilon^0.
$$
\textit{Это верно при $\abs{\alpha} \leqslant \alpha_0$}.

Двигаясь по цепочке $\varepsilon^0 \to \tilde{\varepsilon} \to \varepsilon \to \delta$, возспользуемся оценкой
$$
\norm{\vartheta_h} \leqslant \abs{\alpha} \norm{h} \e^{L\abs{t - \tau}}
$$
для получения
$$
\forall \varepsilon^0 > 0 \exists \tilde{\alpha} > 0 \colon \abs{\alpha} \leqslant \tilde{\alpha} \thus \norm{\vartheta_h} \leqslant \delta.
$$

Это значит, что
$$
\chi_h \rightrightarrows \psi_h \text{ при } \alpha \to 0
$$
на компакте $K = [\tau_0, \tau_1]^2$. Значит, существует искомый предел.

$\lim \chi_h$ --- производная по направлению $h$. Чтобы получить итоговое $Y[\cdot]$, найдём $\dfrac{\d x}{\d \xi_1}, \ldots, \dfrac{\d x}{\d \xi_n}$ \textit{(по стандартному ОНБ)}, и положим столбцы $Y$ равными этим частым производным. Тогда $Y[t]$ будет удовлетворяться матричному \eqref{eq:variation_eq} и начальному условию $T[\tau] = E$.
\end{proof}

\begin{corol}
Чувствительность к вариации в начале:
$$
x(t, \tau, \xi) = x(t, \tau, \xi^0) + Y(t, \tau, \xi^0)(\xi - \xi^0) + \bar{o}(\norm{\xi - \xi^0}).
$$
\end{corol}

\begin{corol}
Когда выполнено условие регулярности из последней теоремы?\\
\textbf{Ответ:} когда $\dfrac{\d f}{\d x}(t,x,u)$ непрерывна по $(t,x,u)$.

Поскольку $y(\cdot) \in L_{\infty}$, $\norm{u(\cdot)} \leqslant C$.
$\forall D \subset \R^n$, $D$ --- компакт, $\forall [\tau_0, \tau_1] \thus K = [\tau_0, \tau_1] \times D \times B_C(0)$ следует, что $\dfrac{\d f}{\d x}$ равномерно непрерывна на $K$ (теорема Кантора), значит
$$
\forall \varepsilon > 0 \exists \delta > 0 : \forall x', x'' \colon \norm{x' - x''} \leqslant \delta \thus
$$
$$
\thus \norm{ \dfrac{\d f}{\d x}(t, x', u) - \dfrac{\d f}{\d x}(t, x'', u) } \leqslant \varepsilon, \;\;\; \forall t \in [\tau_0, \tau_1], \forall u \in B_C(0).
$$
\end{corol}

\begin{corol}
Применение теоремы: $\psi$ --- внешняя нормаль к $\d \mathscr{X}[t]$.
\end{corol}

\end{document}