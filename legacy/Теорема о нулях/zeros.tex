\documentclass[12pt, a4paper]{article}

%\usepackage[cp1251]{inputenc}
\usepackage{a4wide} % уменьшает поля
\textwidth=12cm

\usepackage{background}
\SetBgScale{1}
\SetBgAngle{0}
\SetBgColor{black}
\SetBgContents{\rule{.4pt}{\paperheight}}
\SetBgHshift{5cm}

\usepackage{bm}

\usepackage[utf8]{inputenc}
\usepackage[russian]{babel} % включает русский язык
\usepackage{graphicx} % позволяет подключить .eps - файлы
\usepackage{amsmath}
\usepackage{amsthm} % теоремы от AMS
\usepackage{amssymb} % для работы с математическими R и проч.
\usepackage{floatrow}
\usepackage{mathrsfs}

\usepackage{accents}
\newcommand{\ubar}[1]{\underaccent{\bar}{#1}}

\newtheoremstyle{rusdef}
  {3pt}% measure of space to leave above the theorem. E.g.: 3pt
  {3pt}% measure of space to leave below the theorem. E.g.: 3pt
  {\itshape}% name of font to use in the body of the theorem
  {\parindent}% measure of space to indent
  {\bfseries}% name of head font
  {.}%
  {.5em}%
  {}

\theoremstyle{rusdef}
\newtheorem{definition}{Определение} % определение по-русски
\newtheorem{theorem}{Теорема}
\newtheorem{proposition}{Предложение}
\renewcommand\qedsymbol{$\blacksquare$}
\newtheorem{statement}{Утверждение}
\newtheorem{remark}{Замечание}
\newtheorem{lemma}{Лемма}
\newtheorem{corollary}{Следствие}
\newtheorem{assumption}{Предположение}
\newtheorem{example}{Пример}
\newtheorem{exersize}{Упражнение}

\newcommand\abs[1]{\left\lvert #1 \right\rvert} % модуль
\newcommand\bracket[1]{\left( #1 \right)} % скобки
\newcommand\scalar[1]{\left < #1 \right >} % скалярное произведение
\newcommand{\R}{\ensuremath{\mathbb{R}}} % R - мн-во вещественных чисел
\newcommand{\N}{\ensuremath{\mathbb{N}}} % N - мн-во натуральных чисел
\newcommand{\X}{\mathscr{X}} % красивая Х для начального и конечного множеств
\renewcommand{\P}{\mathscr{P}} % красивая P для ограничений на управление
\newcommand{\then}{\Rightarrow}
\newcommand{\h}{\mathbb{h}}
\newcommand{\e}{\mathbf{e}}

\renewcommand{\H}{\mathcal{H}} % красивая H для Гамильтона-Понтрягина
\newcommand{\M}{\mathcal{M}} % красивая M для Максимума
\renewcommand{\L}{\mathscr{L}} % красивая L для Лагранжа
\renewcommand{\d}{\partial} % чтобы долго не писать частную производную
\newcommand{\norm}[1]{\left\lVert #1 \right\rVert} % норма
\DeclareMathOperator*{\thus}{\Rightarrow} % следствие с возможностью использовать limits
\DeclareMathOperator*{\To}{\longrightarrow}
\DeclareMathOperator*{\Argmax}{Argmax} % Argmax с возмножностью использовать limits

\usepackage{indentfirst} % абзац после заголовка

\begin{document}

\parbox{11.8cm}{
  \begin{center}
    {\Huge Теорема о нулях $x_2$ и $\psi_2$}
  \end{center}
}

\begin{theorem}
  Пусть $\tau_1 < \tau_2$. Тогда
  \begin{enumerate}
    \item если $\psi_2(\tau_1) = \psi_2(\tau_2) = 0$ и $x_2(\tau_1) = 0$, то $x_2(\tau_2) = 0$;
    \item если $x_2(\tau_1) = x_2(\tau_2) = 0$, ${x_2(t) \neq 0}$ при ${t \in (\tau_1, \tau_2)}$ и ${\psi_2(\tau_1) = 0}$, то $\psi_2(\tau_2) = 0$;
    \item если $\psi_2(\tau_1) = \psi_2(\tau_2) = 0$ и $x_2(\tau_1) \neq 0$, то $x_2(\tau_2) \neq 0$ и $\exists t' \in (\tau_1, \tau_2)\colon$ $x_2(\tau') = 0$;
    \item если $x_2(\tau_1) = x_2(\tau_2) = 0$, ${x_2(t) \neq 0}$ при ${t \in (\tau_1, \tau_2)}$ и ${\psi_2(\tau_1) \neq 0}$, то $\psi_2(\tau_2) \neq 0$ и $\exists t' \in (\tau_1, \tau_2)\colon$ $\psi_2(\tau') = 0$.
  \end{enumerate}
\end{theorem}
\begin{proof}
  Докажем все пункты по очереди.
  \begin{enumerate}
    \item[1)] Поскольку $\mathscr{M}|_{t=\tau_1} = \mathscr{M}|_{t=\tau_2}$ из ПМП, то
    $$
      0 = \psi_1(\tau_1) x_2(\tau_1) = \mathscr{M}|_{t=\tau_1} = \mathscr{M}|_{t=\tau_2} = \psi_1(\tau_2) x_2(\tau_2).
    $$
    Поскольку $\psi_1(\tau_2) \neq 0$ в силу (УН), следовательно, $x_2(\tau_2) = 0$.
    
    \item[3)] Аналогично предыдущему пункту: $\psi_1(\tau_1) \neq 0$, $\psi_1(\tau_2) \neq 0$ в силу (УН), и
    $$
      0 \neq \psi_1(\tau_1) x_2(\tau_1) = \mathscr{M}|_{t=\tau_1} = \mathscr{M}|_{t=\tau_2} = \psi_1(\tau_2) x_2(\tau_2).
    $$
    Отсюда $x_2(\tau_2) \neq 0$. Покажем теперь, что в некоторой промежуточной точке $x_2(t') = 0$.
    Без ограничения общности будем считать, что $\tau_1$ и $\tau_2$ --- два последовательных нуля $\psi_2(\cdot)$, и $\psi_2(t) \neq 0$ при $t \in (\tau_1, \tau_2)$.
    Поскольку $\dot{\psi}_2(\tau_j) = -\psi_1(\tau_j) \neq 0$, то 
    %% (TODO: картинка с лекции)
    $$
      \psi_1(\tau_1) \psi_1(\tau_2) < 0.
    $$
    Тогда из равенства $\psi_1(\tau_1) x_2(\tau_1) = \psi_1(\tau_2) x_2(\tau_2)$ следует, что
    $$
      x_2(\tau_1) x_2(\tau_2) < 0,
    $$
    что, в силу непрерывности функции $x_2(\cdot)$, означает, что $\exists t' \in (\tau_1, \tau_2)\colon$ $x_2(t') = 0$.
    \item[2)] Рассмотрим функцию $K(t) = \psi_1(t) x_2(t) + \psi_2(t) \dfrac{dx_2(t)}{dt}$, определённую всюду вне моментов переключений (если $t_0$ --- момент переключения, то производная $\dfrac{dx_2(t)}{dt}$ может иметь разрыв I-го рода). Более того, эта функция будет кусочно-непрерывной.
    Если $t_0$ --- момент переключения, то определены $K(t_0+0)$ и $K(t_0-0)$. Покажем, что $K \equiv const$ (при доопределении в моментах переключений).

    Пусть $\tau$ --- точка непрерывности $K$, тогда при $t \in U_{\delta}(\tau)$:
    $$
      \dfrac{dK}{dt} = \psi_2 \dfrac{\d f}{\d x_1} x_2 + \psi_1(-f + u) + \psi_2 \dfrac{\d f}{\d x_2} \dfrac{dx_2}{dt} - \psi_1 \dfrac{dx_2}{dt} + \psi_2 \dfrac{d^2 x_2}{dt^2}.
    $$
    При этом
    $$
      \psi_2 \dfrac{d^2 x_2}{dt^2} = \psi_2 \left( \dfrac{\d f}{\d x_1} x_2 - \dfrac{\d f}{\d x_2}(-f + u) \right).
    $$
    Объединяя два последних равенства, сократим все слагаемые и получим $\dfrac{dK}{dt} = 0$.

    Пусть теперь $t_0$ --- момент переключения, тогда из ПМП $\psi_2(t_0) = 0$, и
    $$
      K(t_0 \pm 0) = \psi_1(t_0) x_2(t_0) \thus K(t_0 + 0) = K(t_0 - 0), 
    $$
    то есть $K(\cdot)$ непрерывна в $t_0$.

    Тогда доопределим $K$ во всех точках переключений и получим, что $K(t) \equiv const$.
    Тогда ($\psi_2(\tau_1) = 0$):
    $$
      0 = \psi_2(\tau_1) \dfrac{dx_2(\tau_1 \pm 0)}{dt} = K(\tau_1) = K(\tau_2) = \psi_2(\tau_2) \dfrac{dx_2(\tau_2 \pm 0)}{dt}.
    $$
    Предположим противное: пусть $\psi_2(\tau_2) \neq 0$, тогда $\dfrac{dx_2(\tau_2)}{dt} = 0$. Но по условию у нас $x_2(\tau_2)$, и $u(t) \equiv const$ при $t \in U_{\delta}(\tau_2)$. Таким образом, получаем, что $(x_1(\tau_2), x_2(\tau_2))$ --- стационарная точка системы
    $$
      \left\{
        \begin{aligned}
          & \dot{x}_1 = x_2, \\
          & \dot{x}_2 = -f(x_1, x_2) + u.
        \end{aligned}
      \right.
    $$
    В силу единственности решения получаем, что $x_1(t) \equiv const$, $x_2(t) \equiv const$ при $t \in U_{\delta}(\tau_2)$, что противоречит условию $x_2(t) \neq 0$ при $t \in (\tau_1, \tau_2)$.

    Это противоречие приводит нас к тому, что $\psi_2(\tau_2) = 0$.
    \item[4)] По аналогии с 2) через противоречие со стационарной точкой получаем, что при $t \in (\tau_1, \tau_2)$
    $$
        (x_2(t))^2 + \left( \dfrac{dx_2(t \pm 0)}{dt} \right)^2 \neq 0.
    $$
    Тогда 
    $$
      0 \neq \psi_2(\tau_1) \dfrac{dx_2(\tau_1 \pm 0)}{dt} = K(\tau_1) = K(\tau_2) = \psi_2(\tau_2) \dfrac{dx_2(\tau_2 \pm 0)}{dt},
    $$
    то есть $\psi_2(\tau_2) \neq 0$.

    Покажем, что $\exists t' \colon \psi_2(t') = 0$. Предположим противное: пусть $sgn \psi_2(t) \equiv const, t \in [\tau_1, \tau_2]$. Тогда
    $$
        \dfrac{dx_2(\tau_1)}{dt} \dfrac{dx_2(\tau_2)}{dt} > 0,
    $$
    что противоречит (по аналогии с 3)) той части условия 4), в которой говорится, что $\tau_1$ и $\tau_2$ --- последовательные обособленные нули $x_2$.
  \end{enumerate}
  Теорема доказана.
\end{proof}

\end{document}