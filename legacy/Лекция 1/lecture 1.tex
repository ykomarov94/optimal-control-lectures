\documentclass[12pt, a4paper]{article}



%\usepackage[cp1251]{inputenc}
\usepackage{a4wide} % уменьшает поля
\textwidth=11.5cm

\usepackage[utf8]{inputenc}
\usepackage[russian]{babel} % включает русский язык
\usepackage{graphicx} % позволяет подключить .eps - файлы
\usepackage{amsmath}
\usepackage{amsthm} % теоремы от AMS
\usepackage{amssymb} % для работы с математическими R и проч.
\usepackage{floatrow}
\usepackage{mathrsfs}
\usepackage{mathtools} % содержит \coloneqq -- присванивание

\usepackage{accents}
\newcommand{\ubar}[1]{\underaccent{\bar}{#1}}

\usepackage[usenames]{color} % чтобы цвета в hyperref понимались
\usepackage{ulem} % дает зачеркнутый текст по \sout
\usepackage[unicode,
			breaklinks,colorlinks,
			linkcolor=Black,
			citecolor=Black,
			hyperfootnotes=false]
			{hyperref} % кликабельное оглавление без кликабельных футноутов


\newtheoremstyle{rusdef}
  {3pt}% measure of space to leave above the theorem. E.g.: 3pt
  {3pt}% measure of space to leave below the theorem. E.g.: 3pt
  {\itshape}% name of font to use in the body of the theorem
  {\parindent}% measure of space to indent
  {\bfseries}% name of head font
  {.}%
  {.5em}%
  {}
   

\theoremstyle{rusdef}
\newtheorem{define}{Определение} % определение по-русски
\newtheorem{theorem}{Теорема}
\newtheorem{prop}{Предложение}
\renewcommand\qedsymbol{$\blacksquare$}
\newtheorem{statement}{Утверждение}
\newtheorem{remark}{Замечание}
\newtheorem{lemma}{Лемма}
\newtheorem{corol}{Следствие}
\newtheorem{assume}{Предположение}
\newtheorem{example}{Пример}

\newcommand\abs[1]{\left\lvert #1 \right\rvert} % модуль
\newcommand\bracket[1]{\left( #1 \right)} % скобки
\newcommand\scalar[1]{\left < #1 \right >} % скалярное произведение
\newcommand{\R}{\ensuremath{\mathbb{R}}} % R - мн-во вещественных чисел
\newcommand{\N}{\ensuremath{\mathbb{N}}} % N - мн-во натуральных чисел
\newcommand{\X}{\mathscr{X}} % красивая Х для начального и конечного множеств
\renewcommand{\P}{\mathscr{P}} % красивая P для ограничений на управление
\newcommand{\then}{\Rightarrow}
\newcommand{\h}{\mathbb{h}}

\renewcommand{\H}{\mathcal{H}} % красивая H для Гамильтона-Понтрягина
\renewcommand{\d}{\partial} % чтобы долго не писать частную производную
\newcommand{\norm}[1]{\left\lVert #1 \right\rVert} % норма
\DeclareMathOperator*{\thus}{\Rightarrow} % следствие с возможностью использовать limits
\DeclareMathOperator*{\Argmax}{Argmax} % Argmax с возмножностью использовать limits

\usepackage{indentfirst} % абзац после заголовка

\DeclareMathOperator{\sgn}{sgn} % сигнум
\DeclareMathOperator{\rank}{rank} % ранг

\newif\ifproofs
\proofstrue % or
%\proofsfalse

\newif\iflotov
\lotovtrue
%\lotovfalse

\newif\ifnewpages
%\newpagestrue
\newpagesfalse

\newif\iftexts
\textstrue
%\textsfalse

\begin{document}

\begin{center}
{\Huge \textbf{Решение по Каратеодори}}
\end{center}

Рассмотрим систему, движение которой задаётся дифференциальным уравнением
$$
\dot{x}(t) = f(t, x(t), u(t)).
$$
В каком смысле следует понимать траекторию $x(\cdot)$, управление $u(\cdot)$ --- всего лишь измеримая функция? Какие условия при этом необходимо наложить на $f$ и $x$?

Ответу на эти вопросы посвящено это занятие.

Для удобства введем дополнительное обозначение:
$$
g(t,x) = f(t,x,u(t)),
$$
<<спрятав>> таким образом управление внутрь функции $g(t,x)$.

\section*{Условия Каратеодори}
Пусть $(t_0, x^0) \in \R \times \R^n$ и $\exists a>0, r>0$ такие, что:
\begin{enumerate}
\item $g(t,x)$ определена для всех $\forall x \in B_r(x^0)$ и почти всех $\dot{\forall} t \in [t_0 - a, t_0 + a]$;
\item $g(t,x)$ измерима по $t$ для всех $\forall x \in B_r(x^0)$, \\
      $g(t,x)$ непрерывна по $x$ для поти всех $\dot{\forall} t \in [t_0 - a, t_0 + a]$;
\item $\exists m(\cdot)$ - интегрируемая по Лебегу при $t \in [t_0 - a, t_0 + a]$ такая, что:
$$
\norm{g(t,x)} \leqslant m(t), \;\; \forall x \in B_r(x^0), \dot{\forall} t \in [t_0 - a, t_0 + a].
$$
\end{enumerate}
Эти три условия и называются \textit{условиями Каратеодори}

\section*{Абсолютно непрерывные функции}
Итак, мы хотели бы найти решение задачи Коши
\begin{equation}\label{eq:system dotx equal g(t,x(t))}
\left\{
\begin{aligned}
& \dot{x}(t) = g(t, x(t)),\\
& x(t_0) = x^0,
\end{aligned}
\right.
\end{equation}
в следующем классе функций:
\begin{enumerate}
\item $x(\cdot) \in C$;
\item для почти всех $\dot{\forall} t$ существует $\exists \dot{x}$;
\item для почти всех $\dot{\forall} t$ выполнено $\dot{x}(t) = g(t,x(t))$.
\end{enumerate}

Покажем, что условий Каратеодори самих по себе недостаточно для определения решения. Рассмотрим следующий тривиальный пример:
$$
\left\{
\begin{aligned}
& \dot{x} = 0,\\
& x(0) = 0.
\end{aligned}
\right.
$$
Очевидно, решением этой системы будет $x \equiv 0$. Но такое решение в рассматриваемом классе не единственно. \textit{Канторова лестница} также будет являться решением этой системы при всех наложенных ограничениях (если это не очевидно, значит, это \textit{УПРАЖНЕНИЕ}).

Чтобы избежать возникновения подобных артефактов в решениях дифференциальных уравнений, наложим дополнительное условие на $x$:
\begin{center}
$x(\cdot)$ --- решение \eqref{eq:system dotx equal g(t,x(t))} $\Leftrightarrow$ для всех $\forall t$ выполнено
$$
x(t) = x^0 + \int\limits_{t_0}^{t} g(\tau, x(\tau)) d\tau.
$$
\end{center}

Известно, что если $z(\cdot)$ --- измерима, то для любого $\forall \varepsilon > 0$ существует $\exists \delta(\varepsilon) > 0$:
$$
\forall Z \colon \mu{Z} \leqslant \delta \Rightarrow \abs{\int\limits_{\tau \in Z} z(\tau) d\tau} \leqslant \varepsilon,
$$
что означает абсолютную непрерывность интеграла Лебега. Тогда мы можем заменить условие 3) в условиях Каратеодори на следующие два:
\begin{enumerate}
\item[3')] $\dot{x}$ --- интегрируема по Лебегу;
\item[4)] для всех $\forall t \in [t_0 - a, t_0 + a] \Rightarrow x(t) = x^0 + \int\limits_{t_0}^{t} \dot{x}(\tau) d\tau$.
\end{enumerate}

\begin{define}
Функций, удовлетворяющие условиям 1), 2), 3') и 4), будем называть \textbf{абсолютно непрерывными}, а класс таких функций обозначать $AC[t_0 - a, t_0 + a]$.
\end{define}

Класс абсолютно непрерывных функций можно определить иначе.

\begin{define}
Будем говорить, что $x(\cdot) \in AC[\tau_0, \tau_1]$, если для любого $\forall \varepsilon > 0$ существует $\exists \delta(\varepsilon) > 0 \colon$ $\forall \tau_1', \ldots, \tau_k'$, $\tau_1'', \ldots, \tau_k''$ таких, что $\tau_0 \leqslant \tau_1' \leqslant \tau_1'' \leqslant \ldots \leqslant \tau_k' \leqslant \tau_k'' \leqslant \tau_1$ выполнено:
$$
\sum\limits_{j = 1}^{k} \abs{\tau_j'' - \tau_j'} \\leqslant \delta \Rightarrow \sum\limits_{j=1}^{k} \norm{x(\tau_j'') - x(\tau_j')} \leqslant \varepsilon.
$$
\end{define}

\begin{statement}
Определения 1 и 2 эквиваленты
\end{statement}
\begin{proof}
\textit{УПРАЖНЕНИЕ}
\end{proof}

\begin{remark}
Абсолютно непрерывные функции являются непрерывными и равномерно непрерывными, но при этом не обязаны быть дифференцируемыми. В качестве примера можно рассмотреть одномерную функцию $f(x) = \abs{x}$.
\end{remark}

\begin{remark}
$$
Lip[\tau_0, \tau_1] \subseteq AC[\tau_0, \tau_1],
$$
поскольку
$$
\norm{x(\tau'') - x(\tau')} \leqslant L \abs{\tau'' - \tau'} \Rightarrow \delta(\varepsilon) = \frac{\varepsilon}{L}.
$$
\end{remark}

\begin{statement}
Вложение является строгим:
$$
Lip[\tau_0, \tau_1] \subset AC[\tau_0, \tau_1].
$$
\end{statement}
\begin{proof}
\textit{УПРАЖНЕНИЕ}\\
Подсказка: рассмотреть функцию $x(t) = t^{\alpha}, 0 < \alpha < 1$.
\end{proof}

\begin{define}
Решением системы \eqref{eq:system dotx equal g(t,x(t))} на $t_0 - a \leqslant \tau_0 < \tau_1 \leqslant t_0 + a$, $t_0 \in [\tau_0, \tau_1]$ по Каратеодори называется функция $x(\cdot)$, удовлетворяющая следующим критериям:
\begin{enumerate}
\item $x(\cdot) \in AC[\tau_0, \tau_1]$;
\item $x(t_0) = x^0$;
\item для почти всех $\dot{\forall} t \in (\tau_0, \tau_1) \Rightarrow \dot{x}(t) = g(t, x(t))$.
\end{enumerate}
\end{define}

\subsection*{Существование решения по Каратеодори}
Для доказательства основной теоремы нам потребуется сформулировать несколько вспомогательных теорем.

\begin{theorem}[Scorza Dragoni G., 1948]
Пусть $g(t,x)$ --- измерима по $t$ для всех $\forall x \in B_r(x^0)$ и непрерывна по $x$ для почти всех $\dot{\forall} t \in [\tau_0, \tau_1]$. Тогда $\forall \varepsilon >0 \Rightarrow \exists K \subseteq [\tau_0, \tau_1]$, $K$ --- компакт, такой, что
$$
\mu ([\tau_0, \tau_1] \setminus K) \leqslant \varepsilon
$$
и $g \rvert_{K \times B_r(x^0)}$ --- непрерывна по $(t,x)$.
\end{theorem}
\begin{theorem}[Критерий измеримости Лузина]
Функция $z(t)$ --- измерима на $t \in[\tau_0, \tau_1]$ $\Leftrightarrow$ $\forall\,\varepsilon > 0 \;\; \exists K \subseteq [\tau_0, \tau_1]$, $K$ --- компакт, такой, что
$$
\mu([\tau_0,\ \tau_1] \setminus K) \leqslant \varepsilon
$$
и $z \rvert_{K}$ --- непрерывна.
\end{theorem}

\begin{remark}
Из теоремы Лузина следует, что для $g(t,x)$ существует $K(x)$, а из Scorza Dragoni следует существование универсального $K$ (на шаре).
\end{remark}

\begin{corol}[Частный случай Scorza Dragoni]
Если $g(t,x)$ --- измерима по $t$ для всех $\forall x$, непрерывна по $x$ для почти всех $\dot{\forall} t$, а $x(\cdot)$ --- измерима, то функция $g(t, x(t))$ -- измерима по $t$.
\end{corol}
\begin{proof}
Функция $u(\cdot)$ --- измерима, следовательно, из критерия Лузина $\forall \varepsilon > 0 \;\; \exists K \subseteq [t_0 - h, t_0 + h]$, $K$ --- компакт:
$$
\mu([\tau_0,\ \tau_1] \setminus K) \leqslant \varepsilon
$$
и $u \rvert_{K}$ --- непрерывна. Тогда
$$
z(\tau) = g(\tau, x^{(k)}(\tau)) = f(\tau, x^{(k)}(\tau), u(\tau))
$$
непрерывна на $K$, а значит, $z(\cdot)$ --- измерима.
\end{proof}

Теперь мы можем сформулировать основную теорему.

\begin{theorem}[Существование решения \eqref{eq:system dotx equal g(t,x(t))}]
Пусть $0 < h \leqslant a$ и
$$
\int\limits_{t_0}^{t_0 + h} m(\tau) d\tau \leqslant r, \int\limits_{t_0 - h}^{t_0} m(\tau) d\tau \leqslant r.
$$
Тогда существует $\exists x(\cdot) \in AC[t_0 - h, t_0 + h]$ --- решение по Каратеодори системы \eqref{eq:system dotx equal g(t,x(t))}.
\end{theorem}
\begin{proof}
Выпишем следующую последовательность функций:
$$
x^{(0)}(t) \equiv x^0,
$$
$$
x^{(k+1)}(t) = x^0 + \int\limits_{t_0}^{t} g(\tau, x^{(k)}(\tau)) d\tau.
$$

Элементы этой последовательности определены корректно, поскольку $g(\tau, x^{(k)}(\tau))$ измеримы по $\tau$ в силу Следствия 1, ограничены интегрируемой функцией $m(t)$ (по условию теоремы) и, следовательно, интегрируемы по Лебегу. При этом $x^{(k)}(\cdot) \in C$ (более того, $x^{(k)}(\cdot) \in AC$).

Для того, чтобы воспользоваться теоремой Арцела-Асколи, нам необходимо показать равностепенную непрерывность и равномерную ограниченность последовательности.

Равномерная ограниченность (при $t \geqslant t_0$, для $t \leqslant t_0$ --- аналогично):
$$
\norm{x^{(k+1)}(t) - x^0} \leqslant \int\limits_{t_0}^{t} \norm{g(\tau, x^{(k)}(\tau))} d\tau \leqslant \int\limits_{t_0}^{t} m(\tau) d\tau \leqslant r.
$$

Равностепенная непрерывность:
$$
\forall \varepsilon >0 \; \exists \delta(\varepsilon) > 0 \colon \forall t', t'' \in [t_0 - h, t_0 + h], t' \leqslant t'' \colon \abs{t' - t''} \leqslant \delta
$$
$$
\forall n \in \N \Rightarrow \norm{ x^{(n)}(t'') - x^{(n)}(t')} \leqslant \varepsilon?
$$
Для нашей последовательности
$$
\norm{ x^{(n)}(t'') - x^{(n)}(t')} = \norm{\int\limits_{t'}^{t''} g(s, x^{(n-1)}(s))ds} \leqslant\int\limits_{t'}^{t''} m(s) ds \leqslant \varepsilon
$$
в силу абсолютной непрерывности интеграла Лебега.

Тогда последовательность непрерывных функций $\left\{ x^{(k)}(\cdot) \right\}$ равностепенно непрерывно и равномерно ограничено и, в силу теоремы Арцела-Асколи,
$$
x^{(k)} \rightrightarrows x(\cdot).
$$

При этом
$$
\norm{x^{(k)}(\cdot) - x(\cdot)}_C = \max\limits_{t \in [t_0 - h, t_0 + h]} \norm{x^{(k)}(t) - x(t)},
$$
то есть сходимость в $C$ аналогична равномерной сходимости, и $x(\cdot) \in C$.

Наконец, переходим к пределу в итеративной последовательности:
$$
x(t) = x^0 + \int\limits_{t_0}^{t} g(s, x(s)) ds, \;\; x(\cdot) \in AC[t_0 - h, t_0 + h].
$$

Теорема доказана.
\end{proof}

\subsection*{Единственность решения}
Для единственности решения мы обычно требуем липшицевость по $x$:
$$
\norm{g(t,x'') - g(t,x'))} \leqslant L(t) \norm{x'' - x'},
$$
$L(t)$ --- интегрируема по Лебегу.

Ослабив это условие, добавим его к списку \textit{условий Каратеодори} 1)--3):
\begin{itemize}
\item[4')] $\forall x', x'' \;\; \exists L(t)$ --- интегрируемая по Лебегу: 
$$
\scalar{g(t, x'') - g(t,x'), x'' - x'} \leqslant L(t) \norm{x'' - x'}^2.
$$ 
\end{itemize}

Нетрудно показать, что всякая липшицевая по $x$ функция удовлетворяет этому условию в силу неравенства Коши-Буняковского-Шварца.

\begin{theorem}[Теорема о единственности решения по Каратеодори]
Пусть выполнены условия Каратеодори 1), 2), 3), а также условие 4'). Тогда решение по Каратеодори задачи Коши \eqref{eq:system dotx equal g(t,x(t))} единственно.
\end{theorem}
\begin{proof}
Предположим противное. Пусть $x'(t)$ и $x''(t)$ --- два различных решения \eqref{eq:system dotx equal g(t,x(t))} на $[t_0, t_0+h]$.

Рассмотри вспомогательную функцию
$$
z(t) = \norm{x''(t) - x'(t)}^2 = \scalar{x''(t) - x'(t), x''(t) - x'(t)}.
$$
Она дифференцируема почти всюду, и для п.в. $t$
$$
\frac{dz}{dt} = 2 \scalar{g(t, x''(t)), g(t, x'(t)) , x''(t) - x'(t)} \leqslant 2 L(t) z(t).
$$
При этом $z(t_0) = 0$ (из определения $z$).

Тогда неравенство
$$
\frac{dz}{dt} - 2 L(t) z(t) \leqslant 0
$$
домножим на $\exp\left\{-\int\limits_{t_0}^{t} L(\xi) d\xi \right\}$:
$$
\frac{d}{dt} \left( z(t) e^{- 2 \int\limits_{t_0}^{t} L(\xi) d\xi} \right) \leqslant 0
$$
для п.в. $t$ (верно там, где она дифференцируема). Проинтегрировав, получаем:
$$
0 \leqslant z(t) e^{- 2 \int\limits_{t_0}^{t} L(\xi) d\xi} \leqslant 0.
$$
Левое неравенство достигается в силу определния $z$, а правое следует из того факта, что производная отрицательная, а значение $z(t_0) = 0$.

Тогда в обоих случаях достигаются равенства, и функции совпадают.
\end{proof}

\subsection*{Продолжимость решения}
В случае с решением по Каратеодори также возникает вопрос продолжимости решения вправо.\textit{В условиях Каратеодори есть ограниченность интегрируемой функцией, в теореме о существовании решения мы ограничили интеграл от этой функции $m(\cdot)$ значением $r$. Разве этого не достаточно? Оказывается, нет.}

Мы рассматриваем систему на отрезке времени $[t_0 - a, t_0 + a]$. 

Зафиксируем $h_1 < a$ и проинтегрируем исходную систему на $[t_0, t_0 + h_1]$. При этом $\norm{x(t_0 + h_1) - x^0} < r_1$. Переобозначим полученное значение в точке $\xi_1 = x(t_0 + h_1)$ и запишем новую задачу Коши:
$$
\left\{
\begin{aligned}
& \dot{x}(t) = g(t, x(t)), \\
& x(t_0 + h_1) = \xi^1.
\end{aligned}
\right.
$$
Таким образом, мы продвинулись на $h_1$ вправо по времени.

Далее аналогичным образом выберем $h_2, h_3$ и т.~д. Для каждой получающейся задачи Коши мы можем взять новую $m(\cdot)$ и варьировать соответствующее ей значение $r$, устремляя таким образом $h \to a$ и $r \to +\infty$. При этом $r$ не будет ограничено, если $h_1 + h_2 + \ldots < a$. Рассмотрим пример.

\begin{example}
$$
\left\{
\begin{aligned}
& \dot{x}(t) = (x(t))^2, \\
& x(t) = 1
\end{aligned}
\right.
$$
Проинтегрировав систему:
$$
\int \frac{dx}{x^2} = \int dt,
$$
получим решение $x(t) = \frac{1}{1 - t}$, неограниченно растущее в окрестности $t = 1$.
\end{example}

Покажем, что непродолжимость решения может возникать только в случае неограниченного роста функции. Введём обозначения:
$$
\bar{\tau} = \sup \left\{ \tau \in (t_0, t_0 + a) \colon \exists x(\cdot) \text{ --- решение ЗК \eqref{eq:system dotx equal g(t,x(t))} при } t \in [t_0, \tau] \right\},
$$
$$
\ubar{\tau} = \inf \left\{ \tau \in (t_0 - a, t_0) \colon \exists x(\cdot) \text{ --- решение ЗК \eqref{eq:system dotx equal g(t,x(t))} при } t \in [\tau, t_0] \right\}.
$$
Введённые обозначения корректны, поскольку множества непусты в силу существования решения и его ограниченности на отрезке (функции непрерывны).

\begin{theorem}
Пусть $\bar{\tau} < t_0 + a$ ($\ubar{\tau} > t_0 - a$). Тогда для всех $\forall r > 0$ сушествует $\exists \tau \in (t_0, \bar{\tau})$ ($\tau \in (\ubar{\tau}, t_0)$) такое, что $\norm{x(\tau) - x^0} = r$.
\end{theorem}
\begin{proof}
Предположим противное. Пусть $\exists \bar{r} >0$: $\forall \tau \in (t_0, \bar{\tau}) \thus \norm{x(\tau) - x^0} < \bar{r}$.

Пусть $\Delta > 0, r = \bar{r} + \Delta$, тогда $\forall t \in [t_0, \bar{\tau})$ верно
$$
B_{\Delta}(x(t)) \subseteq B_r(x^0).
$$
Возьмём $\delta = t_0 + a - \bar{\tau} > 0$. Тогда $\bar{\tau} + \delta < t_0 + a$.

Для любого $\forall \tau \in [t_0, \bar{\tau}) \thus [\tau - \delta, \tau + \delta] \times B_{\Delta}(x(\tau)) \subseteq [t_0 - a, t_0 + a] \times B_r(x^0)$. 

Существует $\exists h > 0, h < \delta$: $\int\limits_{\tau}^{\tau + h} m(s) ds \leqslant \Delta$. При этом получается, что $h$ --- не зависит от $\tau$ (в силу абсолютной непрерывности интеграла Лебега), то есть мы нашли универсальный шаг, на который можем продвигаться при построении решения: $h$ --- универсально для всех $\tau \in [t_0, \bar{\tau})$, то есть мы можем проинтегрировать $x(\cdot)$ до момента $\tau + h$ для любого $\tau$. По определению $\bar{\tau}$ --- это супремум всех моментов времени, когда существует решение. Из определения супремума: $\exists \tau \colon \bar{\tau} - \tau < h/2$. Для этого $\tau$ и проинтегрируем систему до $\tau + h$. Но тогда получается, что $\tau + h > \bar{\tau}$, что приводит нас к противоречию.

Теорема доказана.
\end{proof}

Отбросим теперь в условиях Каратеодори условие с $a$ и заменим отрезок времени на $[t_0, t_1]$ либо $\R$ (в 1) и 2)) и добавим условия продолжимости вправо(влево)
\begin{equation}
\scalar{g(t,x), x} \leqslant \alpha \norm{x}^2 + \beta \;\; \forall x, \; \alpha, \beta = const > 0
\end{equation}
$$
(-\scalar{g(t,x), x} \leqslant \alpha \norm{x}^2 + \beta).
$$

Условие продолжимости в обе стороны (условие сублинейного роста):
$$
\norm{g(t,x)} \leqslant A \norm{x} + B, \;\; A, B = const > 0.
$$

\begin{remark}
Из условия сублинейного роста следует продолжимость в обе стороны, поскольку
$$
\scalar{g(t,x), x} \leqslant \norm{g(t,x)} \norm{x} \leqslant A \norm{x}^2 + B \norm{x} \leqslant \alpha \norm{x}^2 + \beta.
$$
Как показать, что такие $\alpha, \beta$ существуют? Положим $\alpha = A + 1$, тогда дискриминант $\norm{x}^2 - B\norm{x} + \beta >= 0$ будет отрицательный, то есть это будет верно для всех $\beta$.
\end{remark}

Сфорулируем последнюю на сегодня теорему.

\begin{theorem}
Пусть выполнено условие (2). Тогда решение $x(\cdot)$ задачи Коши \eqref{eq:system dotx equal g(t,x(t))} продолжимо вправо.
\end{theorem}
\begin{proof}
Предположим противное, тогда, в силу предудыщей теоремы, $\norm{x(t)}$ неограничена.

Рассмотрим $z(t) = \norm{x(t)}^2 = \scalar{x(t), x(t)}$.

$$
\frac{dz}{dt} = 2 \scalar{g(t, x(t)), x(t)} \leqslant 2 \alpha z(z) + 2\beta,
$$
$$
\frac{dz}{dt} - 2 \alpha z \leqslant 2\beta.
$$

Домножим на $\exp \left\{ -2\alpha t \right\}$:
$$
\frac{d}{dt}\left( z(t) e^{-2\alpha t} \right) \leqslant \beta e^{-2\alpha t} \thus
$$
$$
\thus z(t) e^{-2\alpha t} - z(t_0) e^{-2\alpha t_0} \leqslant \int\limits_{t_0}^{t} 2 \beta e^{-2 \alpha s} ds \thus
$$
$$
\thus 0 \leqslant z(t) \leqslant z(t_0) e^{-2\alpha t_0} + \int\limits_{t_0}^{t} 2 \beta e^{-2 \alpha s} ds.
$$

Значит, $z(t)$ ограничена, следовательно, $\norm{x}$ ограничена, а значит, продолжимость вправо есть.

Теорема доказана.
\end{proof}

Наконец, можем заменить условие 3) в условиях Каратеодори условием сублинейного роста, положив $m(t) = Ar + B$ ($r$ - из условий теоремы существования решения).

\subsection*{Итоговые условия на $f(t,x,u)$}
\begin{enumerate}
\item $f(t,x,u)$ определена на $\R \times \R^n \times \R^n$ (или $[t_0, t_1] \times \R^n \times \R^n$);
\item $f(t,x,u)$ непрерывна по $(t,x,u)$, $u(\cdot)$ --- измерима;
\item $\norm{f(t,x'',u) - f(t,x',u)} \leqslant L \norm{x'' - x'}$, $L = const$;
\item $\norm{f(t,x,u)} \leqslant A \norm{x} + B$, $\forall(t,x,u)$.
\end{enumerate}
Из них следуют соответствующие \textbf{условия на $g(t,x)$}:
\begin{enumerate}
\item $g(t,x)$ определена при п.в. $t \in \R$ для всех $\forall x$ (п.в. $t \in [t_0, t1]$ для всех $\forall x$);
\item $g(t,x)$ --- измерима по $t$ для всех $x$;\\
	  $g(t,x)$ --- непрерывна по $x$ для п.в. $\dot{\forall} t \in \R (t \in [t_0, t_1])$;
\item $\norm{g(t,x'') - g(t,x')} \leqslant L(t) \norm{x'' - x'}$;
\item условие продолжимости вправо (влево):
$$
\scalar{g(t,x), x} \leqslant \alpha \norm{x}^2 + \beta \;\; \forall x, \; \alpha, \beta = const > 0
$$
$$
(-\scalar{g(t,x), x} \leqslant \alpha \norm{x}^2 + \beta).
$$
\end{enumerate}

\end{document}