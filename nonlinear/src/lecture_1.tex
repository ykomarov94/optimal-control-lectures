\chapter*{Частный случай ПМП}

\textit{Второй семестр посвящён исследованию задач с нелинейной динамикой. Главное отличие таких задач от рассмотренных в прошлом семестре в том, что множества достижимости чаще всего не будут выпуклыми --- и это накладывает ограничения на методы для отыскания решений.}

\textit{Как правило, задачи ОУ для нелинейных систем решают при помощи универсального метода --- Принципа Максимума Л.С.~Понтрягина (ПМП), с которым мы сегодня и начнём своё знакомство.}

\section*{Постановка задачи}
\begin{equation}\label{l1:system}
\left\{
    \begin{aligned}
        & \dot{x}(t) = f(x(t), u(t)), \\
        & u(t) \in \mathscr{P}, \\
        & f = (f^1, \ldots, f^n).
    \end{aligned}
\right.
\end{equation}

В качестве $\mathscr{P}$ можно рассматривать, например,
$$
\begin{aligned}
    & \mathscr{P} = \R^m, \\
    & \mathscr{P} \in \mathrm{conv} \R^m.
\end{aligned}
$$

$$
\begin{aligned}
    & u(\cdot) \text{ --- измерима},
\end{aligned}
$$
\textit{Пока что не затрагиваем вопрос, в каком смысле мы будем понимать траекторию системы \ref{l1:system}. Позже мы формализуем траекторию системы с измеримым управлением с использованием решения по Каратеодори.}
$$
\begin{aligned}
    & t_0, x^0, x^1 \text{ --- фикс.}, \\
    & t_0 \in \R, \\
    & x^0, x^1 \in \R^n.
\end{aligned}
$$

При этом $t_1$ ---  свободно (не фиксировано),
$$
t_1 \in \R, \quad t_1 \geqslant t_0.
$$

Искомое управление должно минимизировать функционал
$$
J = \int\limits_{t_0}^{t_1} f^0(x(t), u(t)) dt \to \inf\limits_{u(\cdot)}.
$$

\section*{Переход к ПМП}
Для того, чтобы записать ПМП для рассматриваемой задачи, <<спрячем>> $J$ внутрь системы \eqref{l1:system}. Введём новую переменную
$$
    x_0(\tau) = \int\limits_{t_0}^{\tau} f^0(x(t), u(t)) dt.
$$

Тогда, переобозначив
$$
\begin{aligned}
    & \tilde{f} = (f^0, f^1, \ldots, f^n)' \\
    & \tilde{x} = (x_0, x_1, \ldots, x_n)'
\end{aligned}
$$
приведём систему к виду $\dot{\tilde{x}} = \tilde{f}(\tilde{x}(t), u(t))$.

Вообще говоря, более точная её запись $\dot{\tilde{x}} = \tilde{f}(x(t), u(t))$, поскольку 
$$
\dot{\tilde{x}} = \tilde{f}(\tilde{x}(t), u(t)) \Leftrightarrow
    \left\{
        \begin{aligned}
            & \dot{x}_0(t) = f^0(x(t), u(t)), \\
            & \dot{x}(t) = f(x(t), u(t)),
        \end{aligned}
    \right.
$$
то есть левая часть зависит только от $x(t)$ и не зависит от $x_0$.

Краевые условия приобретают вид
$$
\tilde{x}(t_0) = \begin{bmatrix} 0 \\ x^0 \end{bmatrix}
\Leftrightarrow
\left\{
    \begin{aligned}
        & x_0(t_0) = 0, \\
        & x(t_0) = x^0.
    \end{aligned}
\right.
$$

Обобщая сказанное, получаем:
\begin{equation}\label{l1:pmp_system}
    \left\{
        \begin{aligned}
            & \dot{\tilde{x}}(t) = \tilde{f}(x(t), u(t)), \\
            & u(t) \in \mathscr{P}, \\
            & \tilde{x}(t_0) = \begin{bmatrix} 0 \\ x^0 \end{bmatrix}, \quad \tilde{x}(t_1) \in \R \times \left\{ x^1 \right\}.
        \end{aligned}
    \right.
\end{equation}

\subsection*{Функция Гамильтона-Понтрягина}
$$
\tilde{\mathscr{H}}(\tilde{\psi}, \tilde{x}, u) \colon \R^{n+1} \times \R^{n+1} \times \R^m \to \R.
$$
\textit{Обратите внимание, что $\tilde{\psi}, \tilde{x}, u$ --- конечномерные аргументы, а не функции!}

Здесь $\tilde{\psi} = (\psi_0,\ \psi_1, \ldots, \psi_n)' = (\psi_0, \psi)'$ --- сопряжённые переменные. По определению
$$
\tilde{\mathscr{H}}(\tilde{\psi}, \! \tilde{x}, \! u) \! = \! \scalar{\tilde{\psi}}{\tilde{f}(\tilde{x},\!u)} \! = \! \scalar{\tilde{\psi}}{\tilde{f}(x,\!u)} \! = \! \psi_0 f^0(x,\!u) + \psi_1 f^1(x,\!u) + \ldots + \psi_n f^n(x,\!u).
$$

\subsection*{Принцип Максимума Понтрягина}

\begin{theorem}[частный случай ПМП]
    Пусть $\{x^*(\cdot), u^*(\cdot)\}$ --- оптимальная пара (при $t_1 = t_1^*, t \in [t_0^*, t_1^*], t_0 = t_0^*$).

    Тогда $\exists \tilde{\psi}^* \colon [t_0^*, t_1^*] \to \R^{n+1}$ такая, что:
    \begin{enumerate}
        \item[(УН) 1)] $\tilde{\psi}^*(\cdot) \not\equiv 0$ ($\Leftrightarrow \tilde{\psi}^*(t) \neq 0$ для п.в. $t \in [t_0^*, t_1^*]$);
        \item[(СС) 2)]
        $$
            \dot{\tilde{\psi}}^*(t) = \left. -\frac{\d \tilde{\mathscr{H}}}{\d \tilde{x}} \right|_{\begin{matrix} \tilde{\psi} = \tilde{\psi}^*(t) \\ \tilde{x} = \tilde{x}^*(t) \\ u = u^*(t) \end{matrix}},
            \quad \text{для п.в.} t \in [t_0^*, t_1^*];
        $$
        \item[(УМ) 3)] $\tilde{\mathscr{H}}(\tilde{\psi}^*(t), \tilde{x}^*(t), u^*(t)) = \sup\limits_{u \in \mathscr{P}} \tilde{\mathscr{H}}(\tilde{\psi}^*(t), \tilde{x}^*(t), u)$ для п.в. $t \in [t_0^*, t_1^*]$.
        \item[4)] $\psi_0^* (\cdot) \equiv const \leqslant 0$, \\
        $\tilde{\mathscr{H}}(\tilde{\psi}^*(t), \tilde{x}^*(t)) \equiv const = 0$.
    \end{enumerate}
\end{theorem}

\subsection*{Обсуждение}

\begin{remark}
    Т.о. при помощи (УМ) ушли от задачи в терминах функций к задаче в точке:
    $$
        u^*(t) \in \Argmax\limits_{u \in \mathscr{P}} \tilde{\mathscr{H}}(\tilde{\psi}^*(t), \tilde{x}^*(t), u).
    $$
     Введём обозначение $\tilde{\mathscr{H}}(\tilde{\psi}, \tilde{x}) = \sup\limits_{u \in \mathscr{P}} \tilde{\mathscr{H}}(\tilde{\psi}, \tilde{x}, u)$.
\end{remark}

\begin{remark}
    Пункт 4) практически полностью следует из предыдущих. Действительно, по построению
    $$
        \dot{\tilde{x}}^*(t) = \left. \frac{\d \tilde{\mathscr{H}}}{\d \tilde{\psi}} \right|_{\begin{matrix} \tilde{\psi} = \tilde{\psi}^*(t) \\ \tilde{x} = \tilde{x}^*(t) \\ u = u^*(t) \end{matrix}},
    $$
    а из (СС) следует, что
    $$
    \frac{d \tilde{\mathscr{H}}}{d t} = \left.\left(\frac{\d \tilde{\mathscr{H}}}{\d \tilde{\psi}} \frac{d \tilde{\psi}}{d t} + \frac{\d \tilde{\mathscr{H}}}{\d \tilde{x}} \frac{d \tilde{x}}{d t} \right)\right|_{\begin{matrix} \tilde{\psi} = \tilde{\psi}^*(t) \\ \tilde{x} = \tilde{x}^*(t) \end{matrix}} = \dot{\tilde{x}}^*(t) \dot{\tilde{\psi}}^*(t) - \dot{\tilde{\psi}}^*(t) \dot{\tilde{x}}^*(t) = 0.
    $$
    % Равенство $\tilde{\mathscr{H}}(\tilde{\psi}^*(t), \tilde{x}^*(t)) = 0$ следует НЕИЗВЕСТНО ОТКУДА TODO.

    Первое условие из 4) следует из того,
    $$
        \dot{\psi}_0^*(t) = \left. -\frac{\d \tilde{\mathscr{H}}}{\d x_0} \right|_{\ldots} = 0.
    $$
    Для остальных же переменных:
    $$
        \dot{\psi}_i^*(t) = \left. -\frac{\d \tilde{\mathscr{H}}}{\d x_i} \right|_{\ldots} = -\psi_0^* \dfrac{\d f^0}{\d x_i} (x^*(t), u^*(t)) - \ldots - -\psi_n^* \dfrac{\d f^n}{\d x_i} (x^*(t), u^*(t)).
    $$
    В векторном виде это
    $$
        (CC') \quad \dot{\tilde{\psi}}^*(t) = - \left( \dfrac{\d \tilde{f}}{\d \tilde{x}} \right)' \tilde{\psi^*}(t),
    $$
    то есть система линейных ОДУ.
\end{remark}

Обсудим теперь условия из ПМП --- почему  мы накладываем такие ограничения?
\begin{enumerate}
    \item Пусть в какой-то момент времени $\tilde{\psi}^*(\tau) = 0$. Поскольку $(CC')$ --- система линейных ОДУ, её решение единственно. Поскольку $\tilde{\psi}(t) \equiv 0$ удовлетворяет $(CC')$ с нулевым начальным условием, то это единственное решение $(CC')$, то есть $\tilde{\psi}(\tau) = 0 \Leftrightarrow \tilde{\psi}(t) \equiv 0$.
    \item Если бы $\tilde{\psi}^* \equiv 0$, то из (СС) и (УМ) мы бы получили тождества $0 = 0$, и  ПМП не помог бы решить задачу.
    \item Функция $\tilde{\psi}^*(\cdot)$ определена с точностью до $\alpha = const > 0$, то есть не единственным образом. Для того, чтобы внести единообразие в поиск решения, в случае $\psi_0^* \neq 0$ \textit{(нормальный случай)} будем брать $\alpha = \dfrac{1}{\abs{\psi_0^*}}$ (т.о. сводя к $\psi_0^* = -1$). В \textit{анормальном случае} ($\psi_0^* = 0$) будем требовать $\norm{\tilde{\psi}^*(t_0)} = 1$.
\end{enumerate}

\begin{remark}
    Функция $\tilde{\mathcal{H}}(\tilde{\psi}, \tilde{x}, u)$ называется функцией Гамильтона-Понтрягина потому, что
    $$
        \left\{
            \begin{aligned}
                & \dot{\tilde{x}}^*(t) = \left. -\dfrac{\d \tilde{\mathcal{H}}}{\d \tilde{\psi}} \right|_{\ldots},\\
                & \dot{\tilde{\psi}}^*(t) = \left. -\dfrac{\d \tilde{\mathcal{H}}}{\d \tilde{x}} \right|_{\ldots}
            \end{aligned}
        \right.
    $$
    при подстановке $u = u^*$ становится Гамильтоновой системой, $\tilde{\mathscr{H}}(\tilde{\psi}, \tilde{x})$ --- гамильтониан.
\end{remark}

\subsection*{Пример того, как пользоваться ПМП}
Пусть $\Argmax\limits_{u \in \mathscr{P}} \tilde{\mathscr{H}}(\tilde{\psi}, \tilde{x}, u) = \left\{u^{\star}(\tilde{\psi}, \tilde{x})\right\}$.

\begin{enumerate}
    \item Ищем все пары $\{x(\cdot), u(\cdot)\}$, совместимые с условиями ПМП. Для этого
    \begin{enumerate}
        \item Если $\{x^*(\cdot), u^*(\cdot)\}$ --- оптимальная пара, то из (УМ) следует 
        $$
            u^*(t) = u^{\star}(\tilde{\psi}^*(t), \tilde{x}^*(t)).
        $$
        \item Выписываем объединённую систему:
        $$
            \left\{
                \begin{aligned}
                    & \dot{\tilde{x}}(t) = \tilde{f}(\tilde{x}(t), u^{\star}(\tilde{\psi}(t) ,\tilde{x}(t))), \\
                    & \dot{\tilde{\psi}}(t) = - \left(\dfrac{\d \tilde{f}(\tilde{x}(t), u^{\star}(\tilde{\psi}(t) ,\tilde{x}(t)))}{\d \tilde{x}}\right)' \tilde{\psi}(t).
                \end{aligned}
            \right.
        $$
        Это система нелинейных ОДУ относительно $\tilde{\psi}, \tilde{x}$. Добавим краевые условия (для нормального случая, анормальный выписывается аналогично):
        $$
            \left\{
                \begin{aligned}
                    & \tilde{x}(t_0) = \tilde{x}^0 = \begin{bmatrix} 0 \\ x_0 \end{bmatrix}, \\
                    & \tilde{\psi}(t_0) = \tilde{\psi}^0 = \begin{bmatrix} -1 \\ \psi^0 \end{bmatrix}.
                \end{aligned}
            \right.
        $$
        Здесь $\psi^0$ --- просто формальное обозначение, мы вводим $n$ новых переменных.
        \item Решаем систему, получаем
        $$
            \left\{
                \begin{aligned}
                    & \tilde{x}[t] = \tilde{x}(t, t_0, \tilde{x}^0, \tilde{\psi^0}), \\
                    & \tilde{\psi}[t] = \tilde{\psi}(t, t_0, \tilde{x}^0, \tilde{\psi}^0).
                \end{aligned}
            \right.
        $$
        \item К этому шагу мы уже ввели $n+1$ переменную:
        $$
            \left\{ \psi^0_1, \ldots, \psi^0_n, t_1 \right\}.
        $$
        У нас есть $n$ условий на правом конце для $x_i[t_1]$, позволяющих их найти:
        $$
            x_i[t_1; t_0, \tilde{x}^0, (-1, \psi^0)'] = x_i^1
        $$
        плюс дополнительное условие из 4):
        $$
            \tilde{\mathscr{H}}(\tilde{\psi}^*(t), x^*(t)) \equiv const = 0.
        $$
        Решая полученную систему, получаем параметризованное управление $u^{\star}(\tilde{x}^0, \tilde{\psi^0})$ и соответствующую ему траекторию.
    \end{enumerate}
    \item Как правило, решений-параметров будет конечное число. Как выбрать одно, решающее исходную задачу?
        \begin{enumerate}
            \item Условия 2-го порядка в ПМП \textit{(мы их использовать не будем)}.
            \item Подставить в $J[\cdot]$ и найти минимизирующие значения параметров.
        \end{enumerate}
    \end{enumerate}

\begin{example}
    $$
        \left\{
            \begin{aligned}
                & \dot{x} = u, \\
                & J = \int\limits_{0}^{t_1} u^2(t) dt \to \min, \\
                & x(0) = 0 \to x(t_1) = 1, \\
                & u(t) \in [-1; 1].
            \end{aligned}
        \right.
    $$
    Выписываем систему на $\tilde{x}$:
    $$
        \left\{
            \begin{aligned}
                & \dot{x}_0(t) = u^2(t), \\
                & \dot{x}_1(t) = u(t),
            \end{aligned}
        \right.
    $$
    отсюда
    $$
        \tilde{\mathscr{H}}(\tilde{\psi}, \tilde{x}, u) = \psi_0 u^2 + \psi_1 u.
    $$

    Нормальный случай: $\psi_0 < 0 \thus \psi_0 = -1$. Ищем максимизаторы для $\tilde{\mathscr{H}}$:
    $$
        \dfrac{d}{du}(-u^2 + \psi_1 u) = -2u + \psi_1 = 0 \thus u^{\star} = \dfrac{\psi_1}{2}.
    $$
    Выписываем $(CC')$:
    $$
        \left\{
            \begin{aligned}
                & \dot{\psi}_0(t) = 0, \\
                & \dot{\psi}_1(t) = - \dfrac{\d \tilde{\mathscr{H}}}{\d \tilde{x}} = 0.
            \end{aligned}
        \right.
    $$
    Тогда объединённая система имеет вид (подставляем $u^{\star}$):
    $$
        \left\{
            \begin{aligned}
                & \dot{x}_0 = \dfrac{(\psi_1)^2}{4}, \\
                & \dot{x}_1 = \dfrac{\psi_1}{2}, \\
                & \dot{\psi}_0 = 0, \\
                & \dot{\psi}_1 = 0.
            \end{aligned}
        \right.
    $$
    Начальные условия:
    $$
        \left\{
            \begin{aligned}
                & x_0(0) = 0, \\
                & x_1(0) = 0, \\
                & \psi_0(0) = -1, \\
                & \psi_1(0) = \psi_1^0.
            \end{aligned}
        \right.
    $$
    Проинтегрировав, получим:
    $$
        \left\{
            \begin{aligned}
                & x_0(t) = \left(\dfrac{\psi_1^0}{2}\right)^2 t, \\
                & x_1(t) = \dfrac{\psi_1^0}{2} t, \\
                & \psi_0 \equiv -1, \\
                & \psi_1 \equiv \psi_1^0.
            \end{aligned}
        \right.
    $$
    Найдём значения $\psi_1^0$ и $t_1$:
    $$
        \left\{
            \begin{aligned}
                & x_1(t_1) = 1 = \dfrac{\psi_1^0}{2} t_1, \\
                & \tilde{\mathscr{H}} = - \left(\dfrac{\psi_1^0}{2}\right)^2 + \dfrac{(\psi_1^0)^2}{2} = \dfrac{(\psi_1^0)^2}{4} = 0.
            \end{aligned}
        \right.
    $$
    Эта система не совместна, следовательно, нормальный случай решения не даёт.

    Анормальный случай --- УПРАЖНЕНИЕ --- также решения не даёт, следовательно, оптимального решения данная задача не имеет.
\end{example}

\begin{remark}
    ПМП является необходимым, но не достаточным условием отыскания оптимума.
\end{remark}