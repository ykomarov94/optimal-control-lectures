\chapter*{Частный случай ПМП}

\textit{Второй семестр посвящён исследованию задач с нелинейной динамикой. Главное отличие таких задач от рассмотренных в прошлом семестре в том, что множества достижимости чаще всего не будут выпуклыми --- и это накладывает ограничения на методы для отыскания решений.}

\textit{Как правило, задачи ОУ для нелинейных систем решают при помощи универсального метода --- Принципа Максимума Л.С.~Понтрягина (ПМП), с которым мы сегодня и начнём своё знакомство.}

\section*{Постановка задачи}
\begin{equation}\label{l1:system}
\left\{
    \begin{aligned}
        & \dot{x}(t) = f(x(t), u(t)), \\
        & u(t) \in \mathscr{P}, \\
        & f = (f^1, \ldots, f^n).
    \end{aligned}
\right.
\end{equation}

В качестве $\mathscr{P}$ можно рассматривать, например,
$$
\begin{aligned}
    & \mathscr{P} = \R^m, \\
    & \mathscr{P} \in \mathrm{conv} \R^m.
\end{aligned}
$$

$$
\begin{aligned}
    & u(\cdot) \text{ --- измерима},
\end{aligned}
$$
\textit{Пока что не затрагиваем вопрос, в каком смысле мы будем понимать траекторию системы \ref{l1:system}. Позже мы формализуем траекторию системы с измеримым управлением с использованием решения по Каратеодори.}
$$
\begin{aligned}
    & t_0, x_0, x_1 \text{ --- фикс.}, \\
    & t_0 \in \R, \\
    & x_0, x_1 \in \R^n.
\end{aligned}
$$

При этом $t_1$ ---  свободно (не фиксировано),
$$
t_1 \in \R, \quad t_1 \geqslant t_0.
$$

Искомое управление должно минимизировать функционал
$$
J = \int\limits_{t_0}^{t_1} f^0(x(t), u(t)) dt \to \inf\limits_{u(\cdot)}.
$$

\section*{Переход к ПМП}
Для того, чтобы записать ПМП для рассматриваемой задачи, <<спрячем>> $J$ внутрь системы \eqref{l1:system}. Введём новую переменную
$$
    x_0(\tau) = \int\limits_{t_0}^{\tau} f^0(x(t), u(t)) dt.
$$

Тогда, переобозначив
$$
\begin{aligned}
    & \tilde{f} = (f^0, f^1, \ldots, f^n)' \\
    & \tilde{x} = (x_0, x_1, \ldots, x_n)'
\end{aligned}
$$
приведём систему к виду $\dot{\tilde{x}} = \tilde{f}(\tilde{x}(t), u(t))$.

Вообще говоря, более точная её запись $\dot{\tilde{x}} = \tilde{f}(x(t), u(t))$, поскольку 
$$
\dot{\tilde{x}} = \tilde{f}(\tilde{x}(t), u(t)) \Leftrightarrow
    \left\{
        \begin{aligned}
            & \dot{x}_0(t) = f_0(x(t), u(t)), \\
            & \dot{x}(t) = f(x(t), u(t)),
        \end{aligned}
    \right.
$$
то есть левая часть зависит только от $x(t)$ и не зависит от $x_0$.

Краевые условия приобретают вид
$$
\tilde{x}(t_0) = \begin{bmatrix} 0 \\ x^0 \end{bmatrix}
\Leftrightarrow
\left\{
    \begin{aligned}
        & x_0(t_0) = 0, \\
        & x(t_0) = x^0.
    \end{aligned}
\right.
$$

Обобщая сказанное, получаем:
\begin{equation}\label{l1:system}
    \left\{
        \begin{aligned}
            & \dot{\tilde{x}}(t) = f(x(t), u(t)), \\
            & u(t) \in \mathscr{P}, \\
            & \tilde{x}(t_0) = \begin{bmatrix} 0 \\ x^0 \end{bmatrix}, \quad \tilde{x}(t_1) \in \R \times \left\{ x^1 \right\}.
        \end{aligned}
    \right.
\end{equation}

\subsection*{Функция Гамильтона-Понтрягина}
$$
\tilde{\mathscr{H}}(\tilde{\psi}, \tilde{x}, u) \colon \R^{n+1} \times \R^{n+1} \times \R^m \to \R.
$$
\textit{Обратите внимание, что $\tilde{\psi}, \tilde{x}, u$ --- конечномерные аргументы, а не функции!}

Здесь $\tilde{\psi} = (\psi_0,\ \psi_1, \ldots, \psi_n)' = (\psi_0, \psi)'$ --- сопряжённые переменные. По определению
$$
\tilde{\mathscr{H}}(\tilde{\psi}, \! \tilde{x}, \! u) \! = \! \scalar{\tilde{\psi}}{\tilde{f}(\tilde{x},\!u)} \! = \! \scalar{\tilde{\psi}}{\tilde{f}(x,\!u)} \! = \! \psi_0 f^0(x,\!u) + \psi_1 f^1(x,\!u) + \ldots + \psi_n f^n(x,\!u).
$$