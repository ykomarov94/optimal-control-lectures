\newcommand\abs[1]{\left\lvert #1 \right\rvert} % модуль
\newcommand\bkt[1]{\left( #1 \right)} % скобки
\newcommand\scalar[2]{\left < #1, #2 \right >} % скалярное произведение
\newcommand{\R}{\ensuremath{\mathbb{R}}} % R - мн-во вещественных чисел
\newcommand{\N}{\ensuremath{\mathbb{N}}} % N - мн-во натуральных чисел
\newcommand{\X}{\mathscr{X}} % красивая Х для начального и конечного множеств

\renewcommand{\d}{\partial} % чтобы долго не писать частную производную
\newcommand{\norm}[1]{\left\lVert #1 \right\rVert} % норма
\DeclareMathOperator*{\thus}{\Rightarrow} % следствие с возможностью использовать limits
% \DeclareMathOperator*{\Argmax}{Argmax} % Argmax с возмножностью использовать limits